\documentclass{patmorin}
\listfiles
\usepackage{pat}
\usepackage{paralist}
\usepackage[OT1]{fontenc}
\usepackage[utf8x]{inputenc}
\usepackage{paralist}
\usepackage{bbm}  % needed for \mathbbm{1}


\usepackage{todonotes}

% etoolbox allows for robust commands that don't need \protect, e.g.
% \newrobustcmd{\onesub}{\mathord{\includegraphics{figs/one-sub}}}
% \subsection{Approximate Voronoi Diagrams in $G^{\onesub}_k$}
\usepackage{etoolbox}


\newenvironment{clmproof}{\noindent\emph{Proof of Claim:}}{\hfill\rule{1ex}{1ex}}

\usepackage[longnamesfirst,numbers,sort&compress]{natbib}

\usepackage[mathlines]{lineno}
\setlength{\linenumbersep}{2em}
% \linenumbers
% \rightlinenumbers
% \linenumbers
\newcommand*\patchAmsMathEnvironmentForLineno[1]{%
 \expandafter\let\csname old#1\expandafter\endcsname\csname #1\endcsname
 \expandafter\let\csname oldend#1\expandafter\endcsname\csname end#1\endcsname
 \renewenvironment{#1}%
    {\linenomath\csname old#1\endcsname}%
    {\csname oldend#1\endcsname\endlinenomath}}%
\newcommand*\patchBothAmsMathEnvironmentsForLineno[1]{%
 \patchAmsMathEnvironmentForLineno{#1}%
 \patchAmsMathEnvironmentForLineno{#1*}}%
\AtBeginDocument{%
\patchBothAmsMathEnvironmentsForLineno{equation}%
\patchBothAmsMathEnvironmentsForLineno{align}%
\patchBothAmsMathEnvironmentsForLineno{flalign}%
\patchBothAmsMathEnvironmentsForLineno{alignat}%
\patchBothAmsMathEnvironmentsForLineno{gather}%
\patchBothAmsMathEnvironmentsForLineno{multline}%
}



% Taken from
% https://tex.stackexchange.com/questions/42726/align-but-show-one-equation-number-at-the-end
\newcommand\numberthis{\addtocounter{equation}{1}\tag{\theequation}}

\definecolor{brightmaroon}{rgb}{0.76, 0.13, 0.28}
\definecolor{linkblue}{rgb}{0, 0.337, 0.227}
\newcommand{\defin}[1]{\emph{\textcolor{brightmaroon}{#1}}}
\makeatletter
\def\mathcolor#1#{\@mathcolor{#1}}
\def\@mathcolor#1#2#3{%
  \protect\leavevmode
  \begingroup
    \color#1{#2}#3%
  \endgroup
}
\makeatother
\newcommand{\mathdefin}[1]{\mathcolor{brightmaroon}{#1}}
% \newcommand{\mathdefin}[1]{\color{brightmaroon}#1}}
\setlength{\parskip}{1ex}

% Document-specific commands and math operators
\DeclareMathOperator{\tw}{tw}
\DeclareMathOperator{\bw}{bw}
\DeclareMathOperator{\evol}{Evol}
\DeclareMathOperator{\ivol}{Ivol}
\DeclareMathOperator{\tvol}{Tvol}


\title{\MakeUppercase{Fan-Partitions of Planar Graphs (and Beyond)
  \newline by Local Sparsification and Volume-Preserving Emeddings}}
\author{TBD}


\date{}


\begin{document}

\maketitle

\begin{abstract}
  We show that every $n$-vertex planar graph is contained in the strong product of a fan and a clique of size $\sqrt{n}\log^2 n$.  Equivalently, every $n$-vertex planar graph $G$ contains a vertex subset $X$ of size $O(\sqrt{n}\log^2 n)$ such that $G-X$ has bandwidth $O(\sqrt{n}\log^2 n)$.  This result holds in the more general setting of product structured graphs, which includes bounded genus graphs and $k$-planar graphs for fixed $k$.
\end{abstract}

\section{Introduction}


Notations: $D$ is local density.  $d$ is a distance function. $\Delta$ will be a ``ball radius''.


\section{Background}

\subsection{Distance Functions and Metric Spaces}

A \defin{distance function} over a set $S$ is any function $d:S^2\to\R$ that satisfies $d(x,x)=0$ for all $x\in S$, $d(x,y)\ge 0$ and $d(x,y)=d(y,x)$ for all distinct $x,y\in S$, and $d(x,z) \le d(x,y)+d(y,z)$ for all distinct $x,y,z\in S$.  A \defin{metric space} $\mathcal{M}:=(S,d)$ consists of a set $S$ and a distance function $d$ over some superset of $S$.  For $x\in S$, the \defin{$r$-ball} centered at $x$ is $\mathdefin{B_{\mathcal{M}}(x,r)}:=\{y\in S:d(x,y)\le r\}$.

For a graph $G$ and any two vertices $v,w\in V(G)$, we use $\mathdefin{d_G(v,w)}$ to denote the minimum number of edges in any path from $v$ to $w$ in $G$, or define $d_G(v,w):=\infty$ if $v$ and $w$ are in different connected components of $G$.  If $G$ is connected, then $d_G$ is a distance function over $V(G)$ so $\mathdefin{\mathcal{M}_G}:=(V(G),d_G)$ is a metric space. Any metric space that can be defined this way is referred to as a \defin{graph metric}.

% In this case we will use the shorthand $B_{G}(x,r):=B_{V(G),d_G}(x,r)$.

Since we will be working frequently with strong products. It is worth noting that, for any two graphs $A$ and $B$, $d_{A\boxtimes B}((x_1,x_2),(y_1,y_2))=\max\{d_A(x_1,y_1),d_B(x_2,y_2)\}$.

We say that $\mathcal{M}$ is \defin{finite} if $S$ is finite.  A finite metric space $\mathcal{M}:=(S,d)$ has \defin{local density} at most $D$ if $|B_\mathcal{M}(x,r)|\le Dr$ for each $r\ge 1$ and each $x\in S$.
% It will be convenient for us to work with a discrete version of local density:  A finite metric space $\mathcal{M}:=(S,d)$ \defin{local density} at most $D$ if $|B_\mathcal{M}(x,r)|\le Dr$ for each \emph{integer} $r\ge 1$ and each $x\in S$.
This definition is consistent with the definition of local density of graphs:  A connected graph $G$ has local density at most $D$ if and only if the metric space $(V(G),d_G)$ has (discrete) local density at most $D$.

A \defin{contraction} of a metric space $\mathcal{M}:=(S,d)$ into a metric space $\mathcal{M'}:=(S',d')$ is a function $\phi:S\to S'$ that satisfies $d'(\phi(x),\phi(y))\le d(x,y)$, for each $x,y\in S$.  For two points $x,y\in\R^\ell$ we let $\mathdefin{d_2(x,y)}$ denotes the Euclidean distance between $x$ and $y$.  A contraction of $(S,d)$ into $(\R^\ell, d_2)$ for some $\ell\ge 1$ is called a \defin{Euclidean contraction}.  For $K\subseteq S$ we abuse notation slightly with the shorthand $\mathdefin{\phi(K)}:=\{\phi(x):x\in K\}$.  When $S\subseteq S'$ and $\phi$ is the identity function we say that $\mathcal{M'}$ is a contraction of $\mathcal{M}$.

\begin{obs}\label{contraction_increases_density}
  If a metric space $\mathcal{M}$ has a contraction into a metric space $\mathcal{M}'$ of local density at most $D$ then $\mathcal{M}$ has local density at most $D$.
\end{obs}

\begin{obs}\label{supergraph_contraction}
  For any connected graph $I$ and any connected subgraph $G$ of $I$, $(V(G),d_I)$ is a contraction of $(V(G),d_G)$.
\end{obs}

\begin{proof}
  From the definitions, it follows that $d_I$, restricted to $V(G)$ is a distance function over $V(G)$, so $(V(G),d_I)$ is a metric space.  Since $G$ is a subgraph of $I$, every path in $G$ is also a path in $I$ so, $d_I(x,y)\le d_G(x,y)$ for each $x,y\in V(G)$.  Therefore $(V(G),d_I)$ is a contraction of $(V(G),d_G)$.
\end{proof}


\subsection{Volume-Preserving Contractions}

For a set $K$ of $k\le L$ points in $\R^L$, the \defin{Euclidean volume}, $\mathdefin{\evol(K)}$, is the $(k-1)$-dimensional volume of the simplex whose vertices are the points in $K$.  For example, if $k=3$, then $\evol(K)$ is the area of the triangle whose vertices are $K$ and that is contained in a plane that contains $K$.

The \defin{ideal volume} of a finite metric space $(K,d)$ is defined as $\mathdefin{\ivol_d(K)}:=\max\{\evol(\phi(K)):\text{$\phi$ is a Euclidean contraction of $(K,d)$}\}$.  A Euclidean contraction $\phi:S\to\R^{\ell}$ of a finite metric space $(S,d)$ is \defin{$(k,\eta)$-volume-preserving} if $\evol(\phi(K))\ge \ivol_d(K)/\eta^{k-1}$ for each $k$-element subset $K$ of $S$.

\citet{feige:approximating} introduces the following definition and theorem as a bridge between ideal volume and Euclidean volume. The \defin{tree volume} of a finite metric space $(K,d)$ is defined as $\mathdefin{\tvol_d(K)}:=\prod_{xy\in E(T)} d(x,y)$ where $T$ is a minimum spanning tree of the weighted complete graph with vertex set $K$ where the weight of any edge $xy$ is equal to $d(x,y)$.

\begin{lem}[{\citet[Theorem~3]{feige:approximating}}]
  Let $(S,d)$ be a finite metric space with $|S|=k$.  Then
  \[
    \ivol_{d}(S) \le \frac{\tvol_d(S)}{(k-1)!} \le 2^{(k-2)/2}\ivol_d(S) \enspace .
  \]
\end{lem}

\subsection{Bandwidth from Local Density and Volume-Preserving Contractions}

The following lemma generalizes \citet[Theorem~10]{feige:approximating} from graph metric to general metric spaces and establishes a critical connection between local density and tree volume.

\begin{lem}[{\citet[Theorem~10]{feige:approximating}}]\label{reciprocal_sum}
  For any $n$-element metric space $\mathcal{M}:=(S,d)$ with local density at most $D$ and any positive integer $k$,
  \[
    \sum_{K\in \binom{V(G)}{k}}\frac{1}{\tvol_{d_G}(K)}\le n(DH_n/2)^{k-1} \enspace ,
  \]
  where $\mathdefin{H_n}:=\sum_{i=1}^n\le 1+\ln n$ is the $n$-th harmonic number.
\end{lem}

\begin{proof}
  First we claim that, for any $x\in S$,
  \begin{equation}
    \sum_{y\in S\setminus\{x\}} \frac{1}{d(x,y)} \le \sum_{i=1}^{n-1}\frac{1}{i/D} = DH_{n-1} < DH_n \enspace . \label{reciprocal_crux}
  \end{equation}
  To see why this is so, let $d_1\le\cdots\le d_{n-1}$ denote the distances of the elements in $S\setminus\{x\}$ from $x$.  For each $i\in\{1,\ldots,n-1\}$, let $\mathdefin{z_i}:=\max\{j: d_j \le i/D\}$. Observe that $z_i \le i$ since, otherwise $B_{(S,i/D)}(x)$ has radius $r:=i/D$ and size $j>i=rD$, contradicting the fact that $(S,d)$ has local density at most $D$.  If $z_i=i$ for each $i\in\{1,\ldots,n-1\}$ then $d_i=i/D$ for each $i\in\{1,\ldots,n-1\}$ and $\sum_{y\in S\setminus\{x\}}=\sum_{i=1}^{n-1}\frac{1}{i/D}$, so there is nothing to prove.  Otherwise, consider the minimum $i$ such that $z_i < i$.  Then $z_i=i-1$ and $d_i > i/D$. By reducing $d_i$ to $i/D$ we increase $\sum_{i=1}^{n-1} 1/d_i$ and increase the minimum $i$ such that $z_i < i$.  Repeating this only increases $\sum_{i=1}^{n-1} 1/d_i$ and finishes with $\sum_{i=1}^{n-1} 1/d_i= H_{n-1}$.

  For a set $K$, let $\Pi(K)$ denote the set of all permutations $\pi:\{1,\ldots,k\}\to K$.
  \citet[Lemma~17]{feige:approximating} shows that, for any $k$-element subset $K$ of $S$,
  \[
    \frac{2^{k-1}}{\tvol(K)} \le \sum_{\pi\in\Pi(K)}\frac{1}{\prod_{i=1}^{k-1}d(\pi(i),\pi(i+1))}
  \]
  Therefore, to prove the lemma it is sufficient to show that
  \begin{equation}
    \sum_{K\in\binom{S}{k}}\sum_{\pi\in\Pi(K)}\frac{1}{\prod_{i=1}^{k-1}d(\pi(i),\pi(i+1))} \le n(DH_n)^{k-1} \enspace .
    \label{volume_sum}
  \end{equation}
  The proof is by induction on $k$.  When $k=1$, the outer sum in \cref{volume_sum} has $\binom{n}{1}=n$ terms, each inner sum has $1!=1$ terms, and the denomiator in each term is an empty product whose value is $1$, by convention.  Therefore, for $k=1$, \cref{volume_sum} asserts that $n \le n$, which is certainly true.  Now assume that \cref{volume_sum} holds for $k-1$.  Then
  \begin{align*}
    & \quad \sum_{K\in\binom{S}{k}}\sum_{\pi\in\Pi(K)}\frac{1}{\prod_{i=1}^{k-1}d(\pi(i),\pi(i+1))} \\
    & = \sum_{K'\in\binom{S}{k-1}}\sum_{\pi\in\Pi(K')}\sum_{y\in S\setminus K'}\frac{1}{\prod_{i=1}^{k-2}d(\pi(i),\pi(i+1))}\cdot\frac{1}{d(\pi(k-1),y)} \\
    & = \sum_{K'\in\binom{S}{k-1}}\sum_{\pi\in\Pi(K')}\frac{1}{\prod_{i=1}^{k-2}d(\pi(i),\pi(i+1))}\cdot\sum_{y\in S\setminus K'}\frac{1}{d(\pi(k-1),y)} \\
    & \le \sum_{K'\in\binom{S}{k-1}}\sum_{\pi\in\Pi(K')}\frac{1}{\prod_{i=1}^{k-2}d(\pi(i),\pi(i+1))}\cdot DH_n & \text{(by \cref{reciprocal_crux})}\\
    & \le n(DH_n)^{k-2}DH_n & \text{(by induction)}\\
    & = n(DH_n)^{k-1} \enspace . & & \qedhere
  \end{align*}
\end{proof}

The following theorem is proved implicitly by \citet{feige:approximating} for graph metrics, although the dependence on the parameters $n$, $D$, $k$, and $\eta$ are not stated explicitly there. Here we give a generalization for metric spaces and make the bound on the bandwidth explicit in terms of the parameters $n$, $D$, and $\eta$.  First we need a definition of bandwidth for metric spaces.  Let $(S,d)$ be a metric space and let $x_1,\ldots,x_n$ be an ordering of $S$.  Then $\bw(x_1,\ldots,x_n):=\max\{j-i:d(x_i,x_j)=1\}$ and $\bw(S,d)$ is the minimum bandwidth taken over all orderings of $S$.  Note that if $S$ is a graph metric, then this definition coincides exactly with the




\begin{thm}\label{volume_density_bandwidth}
  Let $(S,d)$ be a $n$-element metric space with local density at most $D$, minimum-distance $\min_{\{x,y\}\in\binom{S}{2}}d(x,y)=1$, and diameter $\max_{\{x,y\}\in\binom{S}{2}}d(x,y)\le n$.  If $(S,d)$ has a $(k,\eta)$-volume-respecting Euclidean contraction $\phi:S\to\R^L$ with $L\ge 4a\ln n$ then
  \[
    \bw(S,d) \in O((nk\log n)^{1/k}\,Dk\eta\log^{3/2} n) \enspace .
  \]
\end{thm}

% \todo[inline]{Define a version of bandwidth for metric spaces. For a layout $x_1,\ldots,x_n$, $\bw(x_1,\ldots,x_n)=\max\{j-i:d(x_i,x_j)=1\}$. Because paths in graphs are made up of steps of length $1$, this is just the minimum distortion of a Euclidean embedding of $\phi:S\to\Z$.  I wonder if we could strengthen this theorem to show $(S,d)$ has an embedding where $|j-i|\le ?d(x_i,x_j)$?  Probably not!}

\begin{proof}
  Let $r$ be a random unit vector in $\R^L$ and for each $v\in S$, let $\mathdefin{h(v)}:=\langle r,\phi(v)\rangle$ be the inner product of $r$ and $\phi(v)$.  We will order the elements of $S$ as $v_1,\ldots,v_n$ so that $h(v_1)\le \cdots \le h(v_n)$.  To prove an upper bound on $\bw(S,d)$, it suffices to show an upper bound that holds with positive probability on the maximum, over all $vw$ with $d(v,w)=1$, of the number of vertices $x$ such that $h(v)\le h(x)\le h(w)$.

  Consider some pair $v,w\in S$ with $d(v,w)=1$. Since $\phi$ is a contraction, $d_2(\phi(v),\phi(w))\le 1$.  By \cite[Proposition~7]{feige:approximating}, $\Pr(|h(v)-h(w)|\ge \sqrt{4a\ln n/L}) \le n^{-a}$, for any $a>0$. Therefore, with probability at least $1-n^{-a+2}$, $|h(v)-h(w)|\le \sqrt{4a\ln n/L}$ for each pair $v,w\in S$ with $d(v,w)=1$.

  Let  $K:=\{v_1,\ldots,v_k\}$ be a $k$-element subset of $S$. First observe that $\evol(\phi(K)\le n^k$, since $\evol(\phi(K))\le\prod_{i=2}^k d_2(\phi(v_{i-1}),\phi(v_i))\le \prod_{i=2}^k d(v_{i-1},v_i)$.  In particular, $\log\evol(\phi(K))\le k\log n$.  Define $\mathdefin{\ell(K)}:=\max_{v\in K}h(v)-\min_{v\in K} h(v)$.  By \cite[Theorem~9]{feige:approximating} there exists a universal constant $\beta$ such that, for any $c\le 1$,
  \[
      \Pr(\ell(K) \le c)
        < \frac{(\beta L)^{k/2}c^k\max\{1,\log(\evol(\phi(K))\}}{k^k\evol(\phi(K))}
        \le \frac{(\beta L)^{k/2}c^kk\log n}{k^k\evol(\phi(K))} \enspace .
  \]
  In particular,
  \begin{align*}
    \Pr(\ell(K) \le \sqrt{4a\ln n/L})
      & <
      % \frac{(4\beta a\ln n)^{k/2}\max\{1,\log(\evol(\phi(K))\}}{k^k\evol(\phi(K))} \\
      % & <
      \frac{(4\beta a\ln n)^{k/2}\, L\log n}{k^k\evol(\phi(K))} \\
      & \le \frac{(4\beta a\ln n)^{k/2}\eta^{k-1}\, k\log n}{k^k\evol(\phi(K))} \\
      & \le \frac{(4\beta a\ln n)^{k/2}\eta^{k-1}(k-1)!2^{(k-2)/2}\, k\log n}{k^k\tvol_{d}(K)} \\
      & \le \frac{(4\beta a\ln n)^{k/2}\eta^{k-1}2^{(k-2)/2}\,k\log n}{\tvol_{d}(K)} \\
      & \le \frac{\left((8\beta a\ln n)^{1/2}\eta\right)^{k}\,k\log n}{\tvol_{d}(K)} \\
  \end{align*}
  We say that $K\in\binom{S}{k}$ is \defin{bad} if $\ell(K) \le \sqrt{4a\ln n/L}$. Then the expected number of bad sets is at most
  \begin{align}
    \sum_{K\in \binom{S}{k}} \Pr(\text{$K$ is bad})
    & \le \sum_{K\in \binom{S}{k}} \frac{\left((8\beta a\ln n)^{1/2}\eta\right)^{k}\, k\log n}{\tvol_{d}(K)} \notag \\
    & \le \left((8\beta a\ln n)^{1/2}\eta DH_n\right)^{k}\, nk\log n
    & \text{(by \cref{reciprocal_sum})} \notag \\
    & \le \left(O(\eta D\log^{3/2}n)\right)^{k}\, nk\log n \enspace .
    \label{bad_expectation}
  \end{align}
  Let $B$ a maximum cardinality subset of $S$ with $\ell(B)<\sqrt{4a\ln n/L}$.  The vertices in $B$ form $\binom{|B|}{k}$ bad sets. Therefore, by Markov's Inequality, the probability that $\binom{|B|}{k}$ exceeds \eqref{bad_expectation} by a factor of at least $2$ is at most $1/2$.  Therefore, with probability at least $1/2$, $\binom{|B|}{k}\le 2\cdot\eqref{bad_expectation}$, which implies that $|B|\in O((nk\log n)^{1/k}\,Dk\eta\log^{3/2} n)$ with probability at least $1/2$.\footnote{Very roughly, $\binom{|B|}{k}$ is approximated by $(|B|/k)^k$.}  Therefore, with probability at least $1/2-n^{-a+2}$, $\bw_d(x_1,\ldots,x_n)\in O((nk\log n)^{1/k}\,Dk\eta\log^{3/2} n)$.
\end{proof}


\section{Local Sparsification}

In this section we will prove a generalization of the following result:

\begin{lem}\label{planar_sparsifier}
  For each $D\le n$, every $n$-planar graph $G$ contains contains a subset $X$ of $O((n/D)\log n)$ vertices such that $G-X$ has local density at most $n/D$.
\end{lem}

Before continuing, we show that this, combined with the following result of \citet{rao:small}, establishes our results for planar graphs.

\begin{thm}[{\citet{rao:small}}]\label{rao}
  For every $k\le n$, every $n$-vertex planar graph $G$ has a $(k,O(\sqrt{\log n}))$-volume-preserving Euclidean contraction.
\end{thm}

Taking $k=\Theta(\log n)$ and combining \cref{rao,volume_density_bandwidth} yields the following corollary:

\begin{cor}\label{planar_density_bandwidth}
  Every $n$-vertex planar graph $G$ with local density at most $D$ has bandwidth $O(D\log^{3} n)$.
\end{cor}

\begin{thm}\label{fan_partition_planar}
  Every $n$-vertex planar graph $G$ contains a set $X$ of $O(\sqrt{n}\log^2 n)$ vertices such that $G-X$ has bandwidth $O(\sqrt{n}\log^2 n)$.
\end{thm}

\begin{proof}
  Let $G$ be an $n$-vertex planar graph.  By \cref{planar_sparsifier}, $G$ contains a vertex set $X$ of size $O(\sqrt{n}\log^2 n)$ such that $G-X$ has local density at most $D:=\sqrt{n}/\log n$.  By \cref{planar_density_bandwidth}, $G-X$ has bandwidth at most $O(D\log^{3} n)=O(\sqrt{n}\log^2 n)$.
\end{proof}

We now turn to the proof of \cref{planar_sparsifier}.  We work in the more general setting of product structured graphs.


\begin{lem}\label{sparsifier_simple}
  For any graph $H$ of treewidth less than $t$, any path $P$, and any $n$-vertex subgraph $G$ of $H\boxtimes P$ there exists $X\subseteq V(H\boxtimes P)$ with $|X|\le 3?t(n/D)\log_2 n$ such that the metric space $(V(G)\setminus X, d_{H\boxtimes P-X})$ has local density at most $D$.
\end{lem}

\begin{proof}[Proof of \cref{planar_sparsifier} assuming \cref{sparsifier_simple}]
  \citet{ueckerdt.wood.ea:improved} show that, for any planar graph $G$, there exists a graph $H$ of treewidth at most $6$ and a path $P$ such that $G$ is isomorphic to a subgraph of $H\boxtimes P$.  Without loss of generality, we treat $G$ as a subgraph of $H\boxtimes P$.  Apply \cref{sparsifier_simple} to $G$, $H$, and $P$ to obtain a set $X'\subseteq V(H\boxtimes P)$ of size $O((n/D)\log_2 n)$ and let $X:=V(G)\cap X'$.  Then $G-X$ is a subgraph of $H\boxtimes P-X'$ so, by \cref{supergraph_contraction}, $(V(G-X),d_{H\boxtimes P-X'})$ is a contraction of $(V(G-X),d_{G-X})$.  By \cref{contraction_increases_density}, the local density of $(V(G-X),d_{G-X})$ is at most the local density of $(V(G-X),d_{H\boxtimes P-X'})$, which is at most $D$.
\end{proof}

\Cref{sparsifier_simple} is a consequence of the following stronger technical lemma, which says that after removing the vertices in $X$, each component of the subgraph induced by $\ell$ consecutive layers of $H\boxtimes P-X$ contains at most $D\ell$ vertices of $G$.
% This immediately implies \cref{sparsifier_simple} because, for any vertex $v$ of $H\boxtimes P-X$ and any radius $r$,  $B_{H\boxtimes P-X}(v,r)$ is contained in at most $2r+1$ consecutive layers of $H\boxtimes P$, so $B_{H\boxtimes P-X}(v,r)$ contains at most $D(2r+1)\le 3Dr$ vertices of $G$.

\begin{lem}\label{sparsifier}
  For any graph $H$ of treewidth less than $t$, any path $P$, and any $n$-vertex subgraph $G$ of $H\boxtimes P$ there exists $X\subseteq V(H\boxtimes P)$ with $|X|\le ?t(n/D)\log_2 n$ such that each positive integer $\ell$, each subpath $y_1,\ldots,y_\ell$ of $P$, and each component $C$ of $(H\boxtimes P)[V(H)\times \{y_1,\ldots,y_r\}]-X$ satisfies $|V(C)\cap V(G)|\le D\ell$.
\end{lem}

\begin{proof}[Proof of \cref{sparsifier_simple} assuming \cref{sparsifier}]
  For any $v\in V(G)$, there exists a subpath $y_1,\ldots,y_{\ell}$ of $P$ with $\ell\le 2r+1$ such that $B_{H\boxtimes P}(v,r)$ is contained in $(H\boxtimes P)[V(H)\times \{y_1,\ldots,y_{\ell}\}]$.  Apply \cref{sparsifier} to $G$, $H$, $P$ and $D'=D/3$.  Then, for any $v$ and $r$, $B_{(H\boxtimes P)-X}(v,r)$ is contained in a single component $C$ of $(H\boxtimes P)[V(H)\times \{y_1,\ldots,y_{\ell}\}]-X$.  Thus $|V(C)\cap V(G)|\le D'\ell= D'(2r+1)\le 3D'r=D$.
\end{proof}

\begin{proof}[Proof of \cref{sparsifier}]
\end{proof}


\section{\boldmath Volume Preserving Embeddings of Subgraphs of $H\boxtimes P$}

Since a number of other minor-closed and non-minor-closed graph classes have a product stucture similar to that of planar graphs, it would be helpful to   We will prove the following result, which generalizes \cref{rao} from planar graphs to product structured graphs. This immediately generalizes \cref{fan_partition_planar} to all of these other graph classes.

\begin{thm}\label{product_contraction}
  For any $K_{t+1}$-minor-free graph $H$, any path $P$, and any $n$-vertex subgraph $G$ of $H\boxtimes P$, the metric space $(V(G),d_{H\boxtimes P})$ has a $(k,O(\sqrt{\log n}))$-volume-preserving Euclidean contraction.
\end{thm}

The proof of \cref{product_contraction} closely follows Rao's proof of \cref{rao}




\subsection{Decomposing $H$: The Klein-Plotkin-Rao Partition}

First, fix some integer $\Delta \ge 1$.
Consider the following random process that, given a connected graph $H$ constructs a random subset $\textsc{Chop}_{\Delta,t}(H)$ of vertices in $H$. If $t=0$ then $\textsc{Chop}_{\Delta,0}(H):=\emptyset$.  Otherwise, select any vertex $x$ in $H$ and choose a uniformly random $r\in\{0,\ldots,\Delta-1\}$.  Let $R:=\{y\in V(H):d_{H}(x,y)\equiv r\pmod{\Delta}\}$ and let $C_1,\ldots,C_m$ be the connected components of $H-R$.  Then $\textsc{Chop}_{\Delta,t}(H):=R\cup\bigcup_{i=1}^m \textsc{Chop}_{\Delta,t-1}(C_i)$.


The following lemma, with a greater dependence on $t$, is due to \citet{klein.plotkin.ea:excluded}. The version shown here was proved by \citet{fakcharoenphol.talwar:improved}. The presentation here closely follows that of \citet{lee:simpler}.

\begin{lem}[{\citet{klein.plotkin.ea:excluded,fakcharoenphol.talwar:improved}}]\label{component_diameter_h}
  If $H$ is a connected $K_{t+1}$-minor-free connected graph, then each component of $H-\textsc{Chop}_{\Delta,t}(H)$ has diameter at most $?t\Delta$.
\end{lem}


% Say that a vertex $x$ of $H$ is \defin{$\delta$-good} with respect to a set $X\substeq V(H)$ if $d_{H}(x,X)\ge \delta$ and $x$ is \defin{$\delta$-bad} with respect to $X$ otherwise.

\begin{lem}\label{delta_bad_h}
  For any vertex $x$ of $H$ and any integer $\delta\ge 1$, $\Pr(d_H(x, \textsc{Chop}_{\Delta,t}(H)< \delta)\le t\,(2\delta-1)/\Delta$.
\end{lem}

\begin{proof}
  The proof is by induction on $t$.
  If $d_H(x,\textsc{Chop}_{\Delta,t}(H) < \delta)$ then $d(x,R)< \delta$ or, if $t>1$,  $d_H(x,\textsc{Chop}_{\Delta,t-1}(C) < \delta)$ where $C$ is the component of $H-R$ that contains $x$.  Since there are at most $2\delta-1$ choices of $r$ for which $d_H(x,R)<\delta$, $\Pr(d_H(x,R)<\delta) \le (2\delta-1)/\Delta$. For $t=1$, this completes the proof.  For $t>1$, the proof follows from the inductive hypothesis on $\textsc{Chop}_{\Delta,t-1}(C)$ and the union bound.
\end{proof}

\subsection{\boldmath Decomposing $H\boxtimes P$}

Let $G$ be a subgraph of $H\boxtimes P$ where $P:=y_1,\ldots,y_h$. Choose a uniformly random $r_0\in\{0,\ldots,\Delta-1\}$, let $Y_0:=Y_{0,\Delta,t}(P):=\{y_i\in V(P):i\equiv r_0\pmod\Delta\}$ and let $Y:=Y_{\Delta,t}(P):=V(H)\times Y_0$.  Let $P_1,\ldots,P_q$ be the components (each of which is a path) of $P-Y_0$ and let $X_1,\ldots,X_q$ be the results of independent executions of $\textsc{Chop}_{\Delta,t}(H)$ and let $X:=\bigcup_{i=1}^q (X_i\times V(P_i))$.  Finally, let $Z:=Z_{\Delta,t}(H,P):=X\cup Y$.

For some intuition, it is helpful to consider the case when $H$ is a path and $t=1$.  In this case, $H\boxtimes P$ is a grid with diagonal edges.  Starting from row $r\in\{0,\ldots,\Delta-1\}$, the set $Y$ contains one out of every $\Delta$ rows of this grid.  Each of the components $C_1,\ldots,C_m$ of $H\boxtimes P-Y$, except the top-most, $C_1$ and bottom-most, $C_m$ is a grid with $\Delta-1$ rows.  For each component $C_i$, the set $X$ contains one out of every $\Delta$ columns of $C_i$, starting from column $r_i\in\{0,\ldots,\Delta-1\}$. \todo{Add figure}

\begin{lem}\label{component_diameter}
  If $H$ is $K_{t+1}$-minor-free, then each component of $H\boxtimes P-Z$ has diameter at most $?t\Delta$.
\end{lem}

\begin{proof}
  Let $C$ be a component of $H\boxtimes P-Z$.  For any two vertices $x:=(x_1,x_2)$ and $y:=(y_1,y_2)$ of $C$, $d_{C}(x,y) = \max\{d_{H-X_j}(x_1,y_1),d_{P}(x_2,y_2)$ where $X_j$ is the result of running $\textsc{Chop}_{\Delta,t}(H)$. By \cref{component_diameter_h},  $d_{H-X_j}(x_1,y_1)\le ?t\Delta$.  By the choice of $Y_0$, $d_{P}(x,y)<\Delta$.
\end{proof}



\begin{lem}\label{delta_bad_product}
  For any vertex $x$ of $H\boxtimes P$ and any integer $\delta\ge 1$, $\Pr(d_{H\boxtimes P}(x,Z)< \delta)\le (t+1)(2\delta-1)/\Delta$.
\end{lem}

\begin{proof}
  If $d_{H\boxtimes P}(x,Z)<\delta$ then $d_{H\boxtimes P}(x,Y)<\delta$ or $d_{C}(x,Z)<\delta$ where $C$ is the component of $H\boxtimes P-Y$ that contains $x$.  Again, there are at most $2\delta-1$ choices of $r_0$ for which $d_{H\boxtimes P}(x,Y)<\delta$, so $\Pr(d_{H\boxtimes P}(x,Y)<\delta)\le (2\delta-1)/\Delta$.  The proof now follows from the union bound and \cref{delta_bad_h}.
\end{proof}

\Cref{delta_bad_product} with $\delta=1$ yields the following results:

\begin{cor}
  For any $x\in V(H\boxtimes P)$, $\Pr(x\in Z)\le (t+1)/\Delta$.
\end{cor}

\subsection{\boldmath The Mapping $\phi$}

Let $Z:=Z_{\Delta,t}(H,P)$ a subset of $V(H\boxtimes P)$ generated by running the procedure described above.  Let $C_1,\ldots,C_p$ be the components of $H\boxtimes P-Z$ and let $\alpha_1,\ldots,\alpha_p$ be mutually independent uniform real numbers in $[0,1]$ and let $\varphi_Z(x):=(1+\alpha_i)\,d_{H\boxtimes P}(x,Z)$, for each $x\in C_i$ and each $i\in\{1,\ldots,p\}$.  Note that for two vertices $x$ and $y$, $|\varphi_Z(x)-\varphi_Z(y)|\le 2d_{H\boxtimes P}(x,y)$.

\begin{obs}\label{uniform}
  For any vertex $x$ of $H\boxtimes P$, $\varphi_Z(x)$ is uniformly distributed in the real interval $[d_{H\boxtimes P}(x,Z), d_{H\boxtimes P}(x,Z)]$.
\end{obs}

The following observation follows immediately from \cref{component_diameter}:
\begin{obs}\label{independent}
  Let $x_1,\ldots,x_i$ be distinct vertices $V(H\boxtimes P)$ such that $d_{H\boxtimes P}(x_i,\{x_1,\ldots,x_{i-1}\})> ?t\Delta$. Then $\varphi_Z(x_i)$ is independent of $\varphi_Z(x_1),\ldots,\varphi_Z(x_{i-1})$.
\end{obs}


Let $a$ be a constant whose value will be lower-bounded later.  We now define a random function $\phi:V(G)\to\R^{L}$ where $L:=\lfloor 1+\log_2 n\rfloor\cdot\lceil a k\ln n\rceil$. For each $i\in\{0,\ldots,\lfloor \log_2 n\rfloor\}$ and each $j\in\{1,\ldots,\lceil a k\ln n\rceil\}$, let $Z_{i,j}\sim Z_{\Delta,t}(H,P)$ be the random subset of $V(H\boxtimes P)$ obtained by using the procedure described above with parameters $\Delta$ and $t$ with all random choices made independently.  For each $x\in V(H\boxtimes P)$, let $\phi_{i,j}(x)\sim \varphi_{Z_{i,j}}(x)$ again with all choices made independently.  Finally,
\[
   \phi(x) := (\phi_{i,j}(x))_{(i,j)\in \{0,\ldots,\lfloor \log_2 n\rfloor\}\times\{1,\ldots,\lceil a k\ln n\rceil\}} \enspace .
\]
Note that $\phi$ is not a Euclidean contraction: $\phi$ has $L$ coordinates and for two vertices $x$ and $y$, the difference between $\phi_{i,j}(x)$ and $\phi_{i,j}(y)$ can be as much as $2d_{H\boxtimes P}(x,y)$ (but not more).  Thus, $d_2(\phi(x),\phi(y))$ can exceed $d_{H\boxtimes P}(x,y)$ by a factor of up to $2\sqrt{L}$ (but not more).  In a final step, we will divide each coordinate of $\phi$ by $2\sqrt{L}$ but, until then, it is more convenient to work with $\phi$.

The rest of the analysis in this section closely follows \citet{rao:small}, which in turn closely follows \citet{feige:approximating} with modifications needed to work in the metric $(V(G),d_{H\boxtimes P})$.  Nevertheless, we proceed slowly and carefully since our setting is significantly different (we are working in a subgraph $G$ of $H\boxtimes P$ but using the metric space $(V(G),d_{H\boxtimes P})$) and we expect that some readers will not be familiar with some methods introduced by \citet{feige:approximating} that are only sketched by \citet{rao:small}.

We make use of this simple \defin{Chernoff Bound}: For a $\operatorname{binomial}(n,p)$ random variable $B$ and any $0<\delta<1$, $\Pr(B \le np/2) \le \exp(-np/8)$.

% Let $\Gamma_k:=\{(\lambda_1,\ldots,\lambda_k)\in [0,1]^k:\sum_{j=1}^k\lambda_j=1\}$, that is, $\Gamma_k$ is the set of coefficients that can be used to obtain a convex combination of $k$ items.


\begin{lem}\label{crux}
  Fix some function $\lambda:(\R^{L})^{p-1}\to \R^{p-1}$. Let $v_1,\ldots,v_p$ be distinct vertices of $H\boxtimes P$ with $d_{H\boxtimes p}(v_p,\{v_1,\ldots,v_{p-1}\})> q$.  Let $(\lambda_1,\ldots,\lambda_{p-1}):=\lambda(\phi(v_1),\ldots,\phi(v_{p-1}))$ and let $x:=\sum_{j=1}^{p-1}\lambda_j\phi(v_j)$.
  Then, with probability at least $1-n^{-3k}$,
  \[
    d_2(\phi(v_p),x)\ge \frac{q\sqrt{ak\ln  n}}{160?(t+1)} \enspace
  . \]
\end{lem}

\begin{proof}
  Let $i$ be an integer such that $q/2?\le 2^i < q/?$, where $?$ is the constant in \cref{component_diameter_h} and let $\Delta=2^i$.  We will focus on the coordinates $\phi_{i,1},\ldots,\phi_{i,\lceil a k\ln n\rceil}$.  We say that $j\in\{1,\ldots,\lceil a k\ln n\rceil\}$ is \defin{good} if $d_{H\boxtimes P}(v_p,Z_{i,j})\ge \Delta/10(t+1)$.  By \cref{delta_bad_product},  $\Pr(\text{$j$ is good})\ge 1-(t+1)(2\Delta/10(t+1)-1)/\Delta > 3/5$. Let $J:=\{j\in\{1,\ldots,\lceil a k\ln n\rceil\}:\text{$j$ is good}\}$.  Since $Z_{i,1},\ldots,Z_{i,\lceil a k\ln n\rceil}$ are mutually independent, $|J|$ dominates\footnote{We say that a random variable $X$ dominates a random variable $Y$ if $Pr(X\ge x)\ge\Pr(Y\ge x)$ for all $x\in\R$.} a $\operatorname{binomial}(\lceil a k\ln n\rceil,3/5)$ random variable. By the Chernoff Bound, $\Pr(|J|\ge \tfrac{3}{10}\lceil a k\ln n\rceil)\ge 1-\exp(-3ak\ln n/80)$.

  Since $d_{H\boxtimes P}(v_p,\{v_1,\ldots,v_{p-1}\})\ge q$, \cref{component_diameter} implies that the component of $H\boxtimes P-Z_{i,j}$ that contains $v_p$ does not contain any of $v_1,\ldots,v_{p-1}$. Since $\lambda$ is defined entirely by $\phi(v_1),\ldots,\phi(v_{p-1})$,   \cref{uniform,independent} implies that, for $j\in J$, $\phi_{i,j}(v_p)$ is uniformly distributed over an interval of length at least $\Delta/10(t+1)$ and the location of $\phi_{i,j}(v_p)$ in this interval is independent of the corresponding coordinate, $x_{i,j}$ of $x$.
  Therefore, for $j\in J$, $\Pr(|\phi_{i,j}(v_p)-x_{i,j}|\ge \Delta/40(t+1))\ge 1/2$.\footnote{The coordinate $\phi_{i,j}(v_p)$ is uniform over some interval $[a,b]$ of length $b-a\ge \Delta/10(t+1)$ whereas $[x_{i,j}-\Delta/40(t+1),x_{i,j}+\Delta/40(t+1)]$ has length $\Delta/20(t+1)$, so $\Pr(|\phi_{i,j}(v_p)-x_{i,j}|\ge \Delta/40(t+1))\ge (b-a-\Delta/20(t+1))/(b-a)\ge 1/2$.}
  Let $J':=\{j\in J:  |\phi_{i,j}(v_p)-x_{i,j}|\ge \Delta/40(t+1)\}$.  Then $|J'|$ dominates a $\operatorname{binomial}(|J|,1/2)$ random variable.  By the Chernoff Bound (and the union bound), $\Pr(|J'|\ge \tfrac{3}{40}\lceil a k\ln n\rceil)\ge 1-\exp(-3ak\ln n/160)-\exp(-3ak\ln n/80)\ge 1-n^{-3k}$ for all $a\ge 500$, $n\ge 2$, and $k\ge 2$. Therefore,
  \begin{align*}
    d_2(\phi(v_p),x)
    & = \left(\sum_{i'=0}^{\lfloor\log_2 n\rfloor}\sum_{j=1}^{\lceil ak\ln  n\rceil}(\phi_{i',j}(v_p)-x_{i',j})^2\right)^{1/2} \\
    & \ge \left(\sum_{j=1}^{\lceil ak\ln  n\rceil}(\phi_{i,j}(v_p)-x_{i,j})^2\right)^{1/2} \\
    & \ge \left(\sum_{j\in J'}(\Delta/40(t+1))^2\right)^{1/2} \\
    & \ge \left((\Delta/40(t+1))^2\cdot \tfrac{1}{4}\lceil ak\ln  n\rceil\right)^{1/2}
      & \text{(with probability at least $1-n^{-3k}$)} \\
    & = \frac{\Delta\sqrt{\lceil ak\ln  n\rceil}}{80(t+1)} \\
    & \ge \frac{q\sqrt{\lceil ak\ln n\rceil}}{160?(t+1)} \enspace . &
    % & \text{(since $\Delta\ge q/2?$)}\enspace .
    \qedhere
  \end{align*}
\end{proof}

\begin{lem}
  Let $H$ be $K_{t+1}$-minor-free graph with at most $n$ vertices, let $P$ be a path with at most $n$ vertices, and let $\phi:V(H\boxtimes P)\to\R^L$ be the probablistic embedding defined above.  Then, for any $k$-element subset $K$ of $V(H\boxtimes P)$,
  \[
    \Pr\left(\evol(\phi(K)) \ge \frac{\tvol_{d_{H\boxtimes P}}(K)\cdot(2\zeta/3)^{k-1}}{(k-1)!}\right) \ge 1- O(kn^{-k}) \enspace .
  \]
  where $\zeta:=\sqrt{\lceil ak\ln n\rceil}/160?(t+1)$ is the expression that also appears in \cref{crux}.
\end{lem}

\begin{proof}
  The following argument is due to \citet{feige:approximating}.  Let $K$ be a set of $k$ vertices of $G$.  Let $T$ be a minimum spanning tree of the complete graph on $K$ where the weight of an edge $xy$ is $d_{H\boxtimes P}(x,y)$.  Let $x_1,\ldots,x_k$ be an ordering of the vertices in $K$ and $T_1,\ldots,T_k$ be a sequence of trees such that $T_{p}$ is a minimum spanning tree of $x_1,\ldots,x_{p}$ that contains $T_{p-1}$ as a subgraph, for each $p\in\{2,\ldots,k\}$.  That such an ordering and sequence of trees exists follows from the correctness of Prim's Algorithm. For each $p\in\{2,\ldots,k\}$, let $q_p:=d_{H\boxtimes P}(x_p,\{x_1,\ldots,x_{p-1}\})$ be the cost of the unique edge in $E(T_p)\setminus E(T_{p-1})$.  Observe that $\prod_{p=2}^k q_p = \tvol_{d_{H\boxtimes P}}(K)$.

  Let $Q_{p-1}:=\left\{\sum_{i=1}^{p-1}\lambda_i\phi(v_i):(\lambda_1,\ldots,\lambda_{p-1})\in\R^{p-1}\right\}$ be the subspace of $\R^L$ spanned by $\phi(v_1),\ldots,\phi(v_{p-1})$.  Observe that each coordinate $\phi_{i,j}(v_p)$ of $\phi(v_p)$ is at most $2(n-1)$, since $\phi_{i,j}(v_p)=\alpha_{Z_{i,j}}(v_p)\cdot d_{H\boxtimes P}(v_p, Z_{i,j})\le 2(n-1)$. Therefore, $\phi(v_p)$ is contained in a ball $B$ of radius $2(n-1)\sqrt{L}$ around the origin. \citet{feige:approximating} uses these two facts to show $Q_{p-1}\cap B$ can be covered by $\Theta(n^{2k})$ balls, each of radius $q\zeta$, such that, if $\phi(v_p)$ is not contained in any of these balls, then $d_2(\phi(v_p),Q)\ge 2q\zeta/3$.  When this happens, $\evol(\{\phi(v_1),\ldots,\phi(v_p)\})\ge (2q\zeta/3)\cdot\evol(\{\phi(v_1),\ldots,\phi(v_{p-1})\})/(p-1)$.  By \cref{crux} and the union bound, the probability that this happens for each $p\in\{2,\ldots,k\}$ is at least $1-O(kn^{-k})$.  Therefore, with probability at least $1-O(kn^{-k})$,
  \[
    \evol(\phi(K)) \ge \prod_{p=2}^{k}\frac{q_p(2\zeta/3)}{p-1} = \frac{\tvol_{d_{H\boxtimes P}(K)}(2\zeta/3)^{k-1}}{(k-1)!} \enspace . \qedhere
  \]
\end{proof}

We now have all the pieces needed to complete the proof of \cref{product_contraction}.

\begin{proof}[Proof of \cref{product_contraction}]
  Pffffffft
\end{proof}

% ?Let $\phi'(x):=\phi(x)/\sqrt{L}$ for each $x\in V(H\boxtimes P)$.



\bibliographystyle{plainurlnat}
\bibliography{fan-partition}

\end{document}
